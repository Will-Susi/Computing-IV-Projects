\section{ps6: RandWriter}\label{sec:ps6}

\subsection{Overview}\label{sec:ps6:overview} % A discussion of the assignment itself

This project uses a given text and generates new random text based off of it.
It takes the text, a kgram length, and a new text length.
A kgram is a substring of the text of the given length.
Using these many many kgrams, and their occurrences, new text of any length can be generating using the past kgrams probabilities.

\subsection{End Product}\label{sec:ps6:accomplish} % What I accomplished and images/results

Below is the output.
This is a string of random text generated from the text provided by Tom Sawyer.
The kgram length was set to 7.
\bigskip
"CHAPTER VI MONDAY morning campfire.
He was literally hated and the old lady to call on Mrs.
Harper hugged and let's swear again in his brass door-knob"

\subsection{What I Already Knew}\label{sec:ps6:knew} % What was known prior to the assignment

I knew how to parse through text in C++..

\subsection{Design Decisions and Implementations}\label{sec:ps6:decisions} % Important decisions or implementations I made

This project created a RandWriter class which was used to hold the kgrams and k+1grams generated from a text. 
I chose to store the kgrams and k+1grams separately as unordered maps, where the key was a string representing the kgram, and the value was an int representing the number of times the kgram occurred.
The constructor parsed through the text and filled the two maps, while also generating a seed that would be used later for RNG.
As for the new text it goes as follows:
\begin{itemize}
\item 'generate' - This function returns a string of new text given a starting kgram and length of generation.
It loops for the desired length of generation, calls 'kRand' to get a pseudo-random character, and returns a string once all those characters have been generated.
\item 'kRand' - This function returns a character based of a given kgram.
The kgram is first checked to actually exist in the kgram map using the 'freq' function, and if it is, it finds and stores all its associated k+1grams.
It then counts the number of total occurrences of all k+1grams with kgram in them is that a k+1gram with more occurrences has a higher probability of being picked.
A number is then randomly generated based off the RandWrite seed, and a distribution is applied to it, so the random number is between 1 and the total occurrences. 
That random number is then used to pick one of the k+1grams.
That character is then taken and returned as the new random character.
\item 'freq' - Determines if a given kgram or a k+1gram can be found in their respective maps.
The overload for a k+1gram wasn't implemented because of how the internal structure of my RandWriter was made (two separate maps).
\end{itemize}

\subsection{What I Learned}\label{sec:ps6:learned} % What I learned because of the assignment

I learned what kgrams are.
I learned C++'s version of random number generation and how to implement that.
I learned more about maps and how to implement them.

\subsection{Challenges}\label{sec:ps6:challenges} % Challenges along the way and any that went unresolved

There were small challenges along the way, but nothing that became a major issue.

\newpage
\subsection{Codebase}\label{sec:ps6:code} % Code: Makefile, .cpp main, .hpp main, .cpp support, .hpp support, .cpp tests

\bigskip
\title{\large Makefile:}
\lstinputlisting{../ps6/Makefile}
\bigskip
\title{\large main.cpp:}
\lstinputlisting{../ps6/main.cpp}
\bigskip
\title{\large Checkers.cpp:}
\lstinputlisting{../ps6/RandWriter.cpp}
\bigskip
\title{\large Checkers.hpp:}
\lstinputlisting{../ps6/RandWriter.hpp}

\newpage
