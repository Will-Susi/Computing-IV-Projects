\section{ps7: Kronos Log Parsing}\label{sec:ps7}

\subsection{Overview}\label{sec:ps7:overview} % A discussion of the assignment itself

This project parses a log file using regular expressions and outputs a report file.

\subsection{End Product}\label{sec:ps7:accomplish} % What I accomplished and images/results

Below is the output.
This is the end portion of a report file.
It shows the final started and completed boots of a log file.
\bigskip
=== Device boot ===
41684(device5\_intouch.log): 2014-02-03 12:45:46 Boot Start
**** Incomplete boot **** 
\bigskip
=== Device boot ===
41694(device5\_intouch.log): 2014-02-03 14:02:34 Boot Start
41802(device5\_intouch.log): 2014-02-03 14:05:18 Boot Completed
	Boot Time: 164000ms

\subsection{What I Already Knew}\label{sec:ps7:knew} % What was known prior to the assignment

I knew how to parse through an input file in C++.

\subsection{Design Decisions and Implementations}\label{sec:ps7:decisions} % Important decisions or implementations I made

There wasn't a need for a class or an outside file so everything was done in main.
First two regex expressions were made to properly detect start and complete boots.
Each line from the log file was checked against tahese expressions.
When a boot started it was written to the report file with a date and time.
If a boot had already started and another boot was encountered before a complete boot an incomplete boot was written to the report file following the previous boot. 
The new boot was then written to the file with a date and time.
If a completion was encountered following a boot it was written to the file with a date and time.
The boot time between the start and completion was computed via the 'convertTime' function and was also added to the report.
This function took two date/time strings, made them ptime objects, subtracted them, and converted the computed time to milliseconds.
Once the whole log file was parsed, a summary was added to the beginning of the report indicating the total lines read, the total of boots started, and the total of completed boots.

\subsection{What I Learned}\label{sec:ps7:learned} % What I learned because of the assignment

I learned what regular expression were and how useful they are for parsing and finding important parts from large amount of data

\subsection{Challenges}\label{sec:ps7:challenges} % Challenges along the way and any that went unresolved

I had a little trouble at the start of where to start because it felt a bit daunting, but once I started it was pretty simple.
I also had some general trouble with the regex and time/date libraries because the API is really bad and hard to follow,
so some functions used from them took a second to get right.

\newpage
\subsection{Codebase}\label{sec:ps7:code} % Code: Makefile, .cpp main, .hpp main, .cpp support, .hpp support, .cpp tests

\bigskip
\title{\large Makefile:}
\lstinputlisting{../ps7/Makefile}
\bigskip
\title{\large main.cpp:}
\lstinputlisting{../ps7/main.cpp}
\bigskip
\title{\large Checkers.cpp:}
\lstinputlisting{../ps7/main.hpp}

\newpage
